\begin{abstract} %% insira abaixo seu abstract

\selectlanguage{portuguese}	%% para os documentos em Ingl�s com abstract.tex em Portugu�s

\hypertarget{estilo:abstract}{} %% uso para este Guia

The rainforests are the habitat for most part of the known species of plant and terrestrial animals. These ecosystem are under increasing threat everywhere. Over the last decades, millions of hectares of rainforests were losts each year. Despite the proliferation of new remote sensing techniques, the available information about forests' status are limited and sparse. The huge task to protect a fraction of the remaining forests for next generations is out of reach of traditional conservation strategies. A collective action is needed to complement the existing initiatives. This dissertation presents the development, integration and tests of a volunteer sensing moule for a deforestation monitoring system based on citizen science concepts. Citizen science is a term used to designate projects which volunteers, individually or connected, who much of them don't have any specific scientific training, perform or manage related tasks of research activities, such as observation, computation or analysis. Once operational, this computational system will provide to anyone (local residents, volunteers, ONGs, governments, etc), in any place of the world, to monitor forests of selected areas around the globe, almost in real time, using a notebook, a table, or a smartphone connected to Internet.

\selectlanguage{portuguese}	%% para os documentos em Ingl�s com abstract.tex em Portugu�s

\end{abstract}