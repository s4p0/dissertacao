\documentclass[%
%%% PARA ESCOLHER O ESTILO TIRE O SIMBOLO %(COMENT�RIO)
%SemVinculoColorido,
%SemFormatacaoCapitulo,
%SemFolhaAprovacao,
%SemImagens,
%CitacaoNumerica, %% o padr�o � cita��o tipo autor-data
PublicacaoDissOuTese, %% (� tamb�m o "default") com ficha catal. e folha de aprova��o em branco. Caso tenha lista de s�mbolos e lista de siglas e abreviaturas retirar os coment�rios dos arquivos siglas.tex e abreviaturasesiglas.tex. Retirar tamb�m os coment�rios indicados nesse arquivo, nos includes
%PublicacaoArtigoOuRelatorio, %% texto sequencial, sem quebra de p�ginas nem folhas em branco
%PublicacaoProposta, %% igual tese/disserta��o, mas sem ficha catal. e fol. de aprov.
%PublicacaoLivro, %% com cap�tulos
%PublicacaoLivro,SemFormatacaoCapitulo, %% sem cap�tulos
english,portuguese %% para os documentos em Portugu�s com abstract.tex em Ingl�s
%portuguese,english %% para os documentos em Ingl�s com abstract.tex em Portugu�s
%,LogoINPE	%% comentar essa linha para fazer aparecer o logo do Governo
]{tdiinpe}
%]{../../../../../iconet.com.br/banon/2008/03.25.01.19/doc/tdiinpe}

% PARA EXIBIR EM ARIAL TIRAR O COMENT�RIO DAS DUAS LINHAS SEGUINTES
%\renewcommand{\rmdefault}{phv} % Arial
%\renewcommand{\sfdefault}{phv} % Arial

% PARA PUBLICA��ES EM INGL�S:
% renomear o arquivo: abnt-alf.bst para abnt-alfportuguese.bst
% renomear o arquivo: abnt-alfenglish.bst para abnt-alf.bst


%%%%%%%%%%%%%%%%%%%%%%%%%%%%%%%%%%%%%%%%%%%%%
%%% Pacotes j� previamente carregados:      %
%%%%%%%%%%%%%%%%%%%%%%%%%%%%%%%%%%%%%%%%%%%%%%%%%%%%%%%%%%%%%%%%%%%%%%%%
%%% ifthen,calc,graphicx,color,inputenc,babel,hyphenat,array,setspace, %
%%% bigdelim,multirow,supertabular,tabularx,longtable,lastpage,lscape, %
%%% rotate,caption2,amsmath,amssymb,amsthm,subfigure,tocloft,makeidx,  %
%%% eso-pic,calligra,hyperref,ae,fontenc                               %
%%%%%%%%%%%%%%%%%%%%%%%%%%%%%%%%%%%%%%%%%%%%%%%%%%%%%%%%%%%%%%%%%%%%%%%%
%%% insira neste campo, comandos de LaTeX %%%
%%% \usepackage{_exemplo_}
% etc.
%%%%%%%%%%%%%%%%%%%%%%%%%%%%%%%%%%%%%%%%%%%%%

%\watermark{Revis�o No. *00*} %% use o comando \watermark para identificar a vers�o de seu documento
%% comente este comando quando for gerar a vers�o final
\usepackage{rotating}
\usepackage{dsfont}
\usepackage{comment}
\usepackage{blindtext}
%\usepackage{todonotes}
\usepackage{ulem}
\usepackage{enumerate}
\usepackage{booktabs}
\usepackage{lscape}
\usepackage{multicol}
\usepackage{todo}
\usepackage[export]{adjustbox}% http://ctan.org/pkg/adjustbox
%\usepackage{enumitem}
\usepackage{booktabs}
\usepackage{enumitem}
\usepackage{pdflscape}
\usepackage{afterpage}
\usepackage{amssymb}% http://ctan.org/pkg/amssymb
\usepackage{pifont}% http://ctan.org/pkg/pifont
\newcommand{\cmark}{\ding{51}}%
\newcommand{\xmark}{\ding{55}}%
\usepackage{pstricks}

% % JSON SET

\usepackage{bera}% optional: just to have a nice mono-spaced font
\usepackage{listings}
\usepackage{xcolor}

\colorlet{punct}{red!60!black}
\definecolor{background}{HTML}{EEEEEE}
\definecolor{delim}{RGB}{20,105,176}
\colorlet{numb}{magenta!60!black}

\lstdefinelanguage{json}{
    basicstyle=\normalfont\ttfamily,
    numbers=left,
    numberstyle=\scriptsize,
    stepnumber=1,
    numbersep=8pt,
    showstringspaces=false,
    breaklines=true,
    frame=lines,
    backgroundcolor=\color{background},
    literate=
     *{0}{{{\color{numb}0}}}{1}
      {1}{{{\color{numb}1}}}{1}
      {2}{{{\color{numb}2}}}{1}
      {3}{{{\color{numb}3}}}{1}
      {4}{{{\color{numb}4}}}{1}
      {5}{{{\color{numb}5}}}{1}
      {6}{{{\color{numb}6}}}{1}
      {7}{{{\color{numb}7}}}{1}
      {8}{{{\color{numb}8}}}{1}
      {9}{{{\color{numb}9}}}{1}
      {:}{{{\color{punct}{:}}}}{1}
      {,}{{{\color{punct}{,}}}}{1}
      {\{}{{{\color{delim}{\{}}}}{1}
      {\}}{{{\color{delim}{\}}}}}{1}
      {[}{{{\color{delim}{[}}}}{1}
      {]}{{{\color{delim}{]}}}}{1},
}

% % JSON SET

\input{./configuracao} %% fa�a as modifica��es pertinentes no arquivo configuracao.tex

\makeindex  %% n�o alterar, gera INDEX, caso haja algum termo indexado no texto

\begin{document} %% in�cio do documento %% n�o mexer

\maketitle  %% n�o alterar, gera p�ginas obrigat�rias (folha de rosto, ficha catalogr�fica e folha de aprova��o), automaticamente

%%% Comente as linhas opcionais abaixo caso n�o as deseje
\include{./docs/resumo} %% obrigat�rio
\include{./docs/abstract} %% obrigat�rio

\includeListaFiguras %% obrigat�rio caso haja mais de 3 figuras, gerado automaticamente
\includeListaTabelas %% obrigat�rio caso haja mais de 3 tabelas, gerado automaticamente

\include{./docs/abreviaturasesiglas} %% opcional %% altere o arquivo siglaseabreviaturas.tex
\include{./docs/simbolos} %% opcional %% altere o arq	uivo simbolos.tex

\includeSumario  %% obrigat�rio, gerado automaticamente
\inicioIntroducao %% n�o altere este comando
\chapter{Introdu��o}
\label{ch:introducao}
\section{Objetivos}
\label{ch:objetivos}

Esta disserta��o tem como objetivo, formalizar a integra��o dos m�dulos j� desenvolvidos (ver figura \ref{fig:estrutura_atual}), para a definitiva automatiza��o do processo de an�lise de uma nova �rea de interesse, e adicionar � arquitetura existente a capacidade de coletar dados in-situ, caracter�stica principal de um projeto de ci�ncia cidad� do tipo de sensoriamento volunt�rio, utilizando-se de dispositivos m�veis utilizados pelos volunt�rios.

\subsection{Objectivos Espec�ficos}
\label{ch:objetivos:especificos}

Os objetivos espec�ficos desta pesquisa s�o:

\begin{enumerate}
    \item Validar o conceito de sensoriamento volunt�rio.

    \item Propor e desevolver um m�dulo de sensoriamento volunt�rio.

    \item Integrar o m�dulo de sensoriamento volunt�rio junto ao projeto \textit{ForestWatchers}.

\end{enumerate}

\section{Organiza��o do Documento}
\label{ch:organizacao}
Este documento est� organizado da seguinte forma: O Cap�tulo \ref{ch:monitoramento_florestas} � feita a revis�o bibliogr�fica das ferramentas de monitoramento de florestas criado pelo INPE e suas metodologias. Uma revis�o do t�pico de ci�ncia cidad� � abordado no Cap�tulo \ref{ch:ciencia_cidada}, onde � realizada uma compara��o com as pesquisas cient�ficas realizadas por volunt�rios antigamente e nos tempos atuais, conhecida por ciencia cidad� moderna. No Cap�tulo \ref{ch:forestwatchers} � comentado sobre o projeto ForestWatchers, a defini��o do projeto, a metodologia empregada e as aplica��es feitas por este. As metodologias e resultados deste trabalho s�o apresentados no Cap�tulo \ref{ch:metodologia} e \ref{ch:resultados}, respectivamente. Para finalizar, as conclus�es s�o feitas no Cap�tulo \ref{ch:conclusoes}.

\chapter{Monitoramento de Florestas}
\label{ch:monitoramento_florestas}

Por mais de duas d�cadas, o Brasil vem utilizando imagens de sat�lites (ver figura \ref{fig:comparacao_resourcesat_modis}) para realizar o monitoramento da Amaz�nia \cite{Monteiro2008}. Estes sistemas, desenvolvidos pelo INPE, tornaram o Brasil uma refer�ncia mundial na �rea \cite{Tollefson2012a}. \citeonline{Kintisch2007} afirma que esse sistema � motivo de admira��o mundial por ser capaz de informar anualmente as estimativas de taxas de desmatamento na Amaz�nia, al�m de emitir alertas semanais para as autoridades pertinentes. Os principais sistemas utilizados pelo INPE na tarefa de monitorar o desmatamento s�o descritos nas se��es a seguir.

\section{PRODES}
\label{ch:prodes}

Em 1988, o Projeto de Monitoramento do Desmatamento na Amaz�nia Legal (PRODES) foi desenvolvido para fornecer informa��es sobre a din�mica anual do desmatamento de cobertura florestal na Amaz�nia Legal. As estimativas geradas pelo PRODES s�o anuais devido � complexidade e ao detalhamento necess�rios para o c�lculo da �rea de desmate. Essas estimativas se baseiam em mapeamento detalhado com um grande conjunto de imagens do tipo LANDSAT (ou equivalente) , que cobrem a Amaz�nia com baixa frequ�ncia temporal (16 e 26 dias, ver figura \ref{fig:swath_width_LANDSAT-NASA}) e com resolu��o��o espacial entre 20 e 30 metros. Esses sensores s�o capazes de mapear desmatamentos cujas �reas sejam superiores a $6,25$ hectares \cite{Monteiro2008}.

\begin{figure}[htb]
    \centering
    \includegraphics[width=\textwidth,height=\textheight,keepaspectratio]{figuras/swath_width_LANDSAT-NASA.jpg}
    \caption{\textit{Swath} do LANDSAT, largura da cobertura di�ria realizado pelo LANDSAT, ilustrado no continente Norte Americano. Para obter toda a extens�o amaz�nica � necess�rio de 16 a 26 dias, dependendo das condi��es clim�ticas do local.  }
    \FONTE{\citeonline{NASA}}
    \label{fig:swath_width_LANDSAT-NASA}
\end{figure}

\subsection{Metodologia}
\label{ch:prodes_metodo}

Para realizar o c�lculo da taxa de desmatamento as imagens s�o selecionadas de modo a obter a menor cobertura de nuvens poss�vel, melhor visibilidade com uma adequada qualidade radiom�trica\footnote{A resolu��o radiom�trica � dada pelo n�mero de valores digitais representando n�veis de cinza, usados para expressar os dados coletados pelo sensor. Quanto maior o n�mero de valores, maior � a resolu��o radiom�trica} e com a data de aquisi��o das imagens pr�xima ao per�odo de refer�ncia para o c�lculo da taxa de desmatamento. Por�m, considerando o hist�rico climatol�gico da Amaz�nia, a maioria das imagens n�o se apresentam livres de nuvens. Por isso � poss�vel utilizar mais de uma imagem (inclusive de outros sat�lites) para compor as cenas, formando um mos�ico (ver figura \ref{fig:prodes_mosaico}). 

\begin{figure}[htb]
    \centering
    \includegraphics[width=\textwidth,height=\textheight,keepaspectratio]{figuras/mosaico_landsat-prodes.png}
    \caption{ Mosa�co formado por imagens LANDSAT a serem utilizadas no sistema PRODES.}
    \FONTE{\citeonline{INPE}}
    \label{fig:prodes_mosaico}
\end{figure}

Ap�s a sele��o das imagens, a pr�xima etapa envolve transformar seus dados radiom�tricos em componentes de cena (vegeta��o, solo e sombra), utilizando o Modelo Linear de Mistura Espectral (MLME). As bandas 3, 4 e 5 do sensor TM s�o utilizadas para estimar a propor��o dos componentes solo, vegeta��o e sombra para cada pixel, formando um sistema de equa��es lineares que pode ser solucionado pelo m�todo dos m�nimos quadrados ponderados. O resultado desse modelo linear � uma imagem-fra��o, onde se tem tr�s bandas sint�ticas com os valores proporcionais de vegeta��o, solo e sombra. A segmenta��o\footnote{Segmenta��o de imagem � uma t�cnica de agrupamentos de dados onde caracter�sticas espectrais semelhantes s�o agrupadas.} da imagem-fra��o � ent�o realizada, ajustando-se os limiares de similaridades e de �rea. 

Um algoritmo de classifica��o n�o-supervisionado de agrupamentos de dados trata as imagens segmentadas, classificando-as de acordo com as classes definidas pelo banco de dados. Como resultado tem-se uma nova imagem \textit{raster} ou vetorial com as �reas desflorestadas. Ent�o, um fotointerprete tem a tarefa de analisar os pol�gonos tem�ticos gerados, tomando a decis�o se esses devem ser aceitos ou reclassificados. Uma vez essa imagem aceita, uma m�scara de desmatamento contendo as �reas de corte raso j� detectados � gerada. Essa m�scara ser� utilizada para eliminar desmatamentos antigos, impedindo que sejam identificados novamente.

\section{DETER}
\label{ch:deter}\ref{}

Devido ao tempo necess�rio para gerar os resultados e por observar apenas �reas de corte raso, o PRODES n�o pode ser utilizado como um sistema de preven��o. Portanto, a partir de 2004 o Sistema de Detec��o de Desmatamento em Tempo Real (DETER) foi implementado para realizar o monitoramento cont�nuo do desmatamento e da degrada��o florestal. Esse sistema foi criado para atender ao Governo Federal no Plano de A��o para a Preven��o e Controle do Desmatamento na Amaz�nia Legal. O principal objetivo desse sistema � de fornecer informa��es sobre o local e a dimens�o aproximada de ocorr�ncias de mudan�as na vegeta��o de modo a agilizar a fiscaliza��o. 

\subsection{Metodologia}
\label{ch:deter_metodo}

As imagens utilizadas por esse sistema s�o obtidas pelo sensor MODIS (a bordo do sat�lite TERRA da NASA), que cobre a Amaz�nia a cada dia e meio (ver figura \ref{fig:swath_width_MODIS-TERRA}. Essa alta resolu��o temporal reduz as limita��es de observa��o impostas pela cobertura de nuvens da regi�o. Com a m�xima resolu��o espacial limitada em $250$ metros, as imagens desses sensores permitem a detec��o de desmatamentos apenas para �reas maiores do que $25$ hectares. O objetivo do DETER � de fornecer indicadores para fiscaliza��o a cada 15 dias, quando as condi��es de observa��o s�o favor�veis. Esse sistema observa diversos est�gios de desmatamento para emitir seus alertas, como o de corte raso, degrada��o florestal de intensidade alta, m�dia e baixa, sendo o �ltimo mais dif�cil devido a resolu��o das imagens do sensor MODIS \cite{Monteiro2008}.

\begin{figure}[htb]
    \centering
    \includegraphics[width=\textwidth,height=\textheight,keepaspectratio]{figuras/swath_width_MODIS-TERRA.jpg}
    \caption{\textit{Swath} do MODIS, largura da cobertura di�ria obtida a cada sobrevoo. Para obter toda a extens�o amaz�nica � necess�rio em m�dia um dia e meio.}
    \FONTE{\citeonline{NASA}}
    \label{fig:swath_width_MODIS-TERRA}
\end{figure}

A aquisi��o das imagens � feita de forma r�pida, uma vez que que o DETER utiliza os produtos baseados em \textit{granules}\footnote{Granules s�o produtos gerados de uma �rea particular. Granules n�o cobrem todo o globo.} dos subconjuntos de resposta r�pida da NASA. Esses dados encontram-se prontos para serem utilizados, pois j� foram processados, disponibilizados em GeoTIFF, RGB-equivalente, no formato de 8-bits e geograficamente projetados. 

Essas imagens s�o ent�o carregadas no Sistema de Processamento de Informa��es Geo-referenciadas (SPRING) para que outros processamentos sejam feitos. Nesta etapa o especialista necessita aplicar um modelo de mistura para separar o que � floresta, solo ou �gua (ou sombra). Essa etapa � feita selecionando-se certos \textit{pixels} com uma resposta espectral particular. Ent�o, cada imagem � segmentada e classificada. Ap�s a classifica��o das imagens, o especialista aplica as m�scaras dos desflorestamentos anteriores e de hidrografia, com a finalidade de esconder os desmatamentos j� conhecidos assim como outras caracter�sticas. 

Na �ltima etapa, o especialista corrige os resultados da segmenta��o autom�tica, pixel-a-pixel. As vezes � poss�vel que as etapas de classifica��o e segmenta��o possam ser colocados de lado, pois o especialista pode extrair todas as informa��es baseando-se apenas em sua expertise olhando para as imagens do sat�lite, munido dos arquivos geogr�ficos de desflorestamento e hidrografia.

\begin{figure}[htb]
\includegraphics[width=0.485\textwidth]{figuras/COMPARE_RESOURCESAT.png}
\hfill
\includegraphics[width=0.485\textwidth]{figuras/COMPARE_MODIS.png}
\caption{Compara��o de diferentes resolu��es espaciais. � esquerda, uma imagem RESOURCESAT com 23,5 m por pixel. � direita, uma imagem MODIS com 250 m por pixel.}
\label{fig:comparacao_resourcesat_modis}
\end{figure}



\chapter{Ci�ncia Cidad�}
\label{ch:ciencia_cidada}

Ci�ncia cidad� � o termo usado para designar projetos no qual volunt�rios, muitos sem nenhum treinamento cient�fico espec�fico, efetuam ou gerenciam tarefas cient�ficas, tais como a realiza��o de observa��es, medi��es ou computa��o \cite{SoaresSant:2011:EnPoAt}. 

Atrav�s do voluntariado de pessoas ordin�rias, conhecidas como cientistas cidad�o, projetos cient�ficos conseguem obter um quadro maior de colaboradores \cite{Cohn.2008}. Um fator importante em projetos grandes, agilizando o processo de aquisi��o e divulga��o dos resultados. Segundo \citeonline{Silvertown2009}, o cientista cidad�o � um volunt�rio que coleta ou processa dados como parte de uma investiga��o cient�fica. Cientistas cidad�o n�o s�o respons�veis, necessariamente, por analisar ou publicar artigos cient�ficos, estes desempenham tarefas simples, mas de grande import�ncia para a conclus�o dos trabalhos cient�ficos. Nas �ltimas d�cadas, projetos cient�ficos baseados em ci�ncia cidad� ganharam notoriedade, por�m esta abordagem n�o � nova para a comunidade cient�fica. 

Realizar pesquisas cientificas utilizando-se da colabora��o de diversos indiv�duos vem de tempos remotos. Para ilustrar, A Origem das Esp�cies � um exemplo de pesquisa cient�fica realizada com a ajuda de diversos colaboradores j� em 1830. \citeonline{darwin-origin-of-species-1859}, contou com a ajuda de mais de 2000 colaboradores entre esses, especialistas, bi�logos e pesquisadores de diferentes �reas, tendo tamb�m a colabora��o de cientistas de diferentes �reas.  O projeto \textit{\textbf{Darwin Correspondences}}\footnote{\url{http://www.darwinproject.ac.uk/}} reune mais de 7.500 das cartas que Darwin manteve com seus correspondentes durante sua pesquisa \cite{DarwinProject_Correspondents}. 

Os conte�dos destas cartas variavam de nota��es cient�ficas sobre algumas esp�cies, o que requeria um aparato profissional, ou de apenas simples observa��es, que vieram a colaborar com teoria da evolu��o das esp�cies. Naquela �poca, as cartas demoravam meses para serem recebidas e lidas,  o processo de responder uma carta e obter um novo retorno da mesma pessoa, chegava a levar quest�es de anos para acontecer. Tudo isto devido a este meio de comunica��o n�o ter as mesmas tecnologias atuais, tornando a tarefa de entregar uma carta hoje em dia simples e comum , um trabalho dificultoso e lento. Naquela �poca, os servi�os de correspond�ncias era feito por mensageiros a p�, a cavalo ou atrav�s de charretes, tendo n�vios � vela para correspond�ncias pelos mares. \citeonline{Hyde1891} relembra que para a determinada �poca, estes servi�os eram caros e n�o acess�vel para todos, o que dificultava ainda mais o compartilhamento de ideias. Em 1840, com a grande reforma brit�nica de postagem,  \textit{Penny Postage} \footnote{A reforma realizada pela \textit{Royal Mail} do Reino Unido cobrava apenas um \textit{Penny}, menor moeda do sistema monet�rio da �poca, para entregar as cartas indiferente da dist�ncia.}, houve uma maior difus�o e uso dos servi�os, elevando o envio de cartas de 82.500.000 para 169.000.000 em um ano, mais que o dobro.

\citeonline{Zimmer2011}, comenta que os resultados obtidos no recente trabalho de \citeonline{Silvertown2011} seria uma das formas que Darwin faria ci�ncia hoje atrav�s da Internet. Neste trabalho, \citeonline{Silvertown2011} observa mudan�as evolucion�rias de um continente atrav�s de colaboradores que utilizaram um projeto de ci�ncia cidad� moderno, Evolution MegaLab, iniciado em 2008 o projeto contava com colaboradores para enviar informa��es de duas esp�cies de caramujos, Cepaea nemoralis and C. hortensis, para realizar um estudo de evolu��o da esp�cie observando as diferentes cores de suas cascas.

Um projeto da universidade de Oxford ir� investigar o envolvimento do p�blico na ci�ncia do s�culo 19 e 21 \cite{conscicom2014}, receber� o financeamento de quase 2 milh�es de libras para realizar seus estudos. Sup�e-se que este estudo poder� entender e desenvolver novas ferramentas para trocar informa��es entre cientistas profissionais e legi�es de volunt�rios \cite{Leicester2013}. 

\section{Ci�ncia Cidad� Moderna}
\label{ch:ciencia_cidada_moderna}

O projeto considerado um dos primeiros de ci�ncia cidad� moderno � o \textit{Christmas Bird Count}. Um projeto antigo e que ainda encontra-se em atividade, o Christmas Bird Count procura contar as diferentes esp�cies que existem na am�rica do Norte, suas eventuais mudan�as de habitat, entre outras informa��es. Idealizado em 1900 por Frank Chapman, um famoso ornit�logo do Museu Americano de Hist�ria Natural, como uma atividade alternativa ao evento de ca�a aos p�ssaros existente na �poca, Chapman publicou diversos livros com os resultados obtidos por este projeto com a ajuda de milhares de volunt�rios. Estes seguem diversas regras para conduzir a pesquisa durante os 20 dias em que as observa��es s�o feitas, de 14 de dezembro a 5 de janeiro de cada ano, s� podem ser contabilizados os p�ssaros que forem avistados ou ouvidos em um di�metro de 24-km, moradores pr�ximos a estas �reas podem utilizar bebedouros para p�ssaros para atrair mais esp�cies e contabiliz�-los \cite{Silvertown2009}. Em uma contagem recente milhares de observadores relataram mais de $63$ milh�es de p�ssaros. 

Hoje, o centro de pesquisa que deu origem a este projeto � considerado um dos maiores centros especializados em ci�ncia cidad� com diversos estudos em biologia. Cientistas do laborat�rio Cornell de Ornitologia universidade de Cornell, l�dereres no estudo e conserva��o dos p�ssaros, rastreiam projetos que utilizam ci�ncia cidad� para realizar seus estudos. Estes acreditam que trabalhar com cientistas cidad�os � um fen�meno em expans�o em todo o mundo \cite{Cohn.2008}. Este laborat�rio conta com uma comunidade de aproximadamente 200 mil participantes de ci�ncia cidad�.

Ainda h� d�vidas se os projetos de ci�ncia cidad� possam gerar resultados confi�veis, uma vez que muitos dos volunt�rios engajados nas atividades cient�ficas n�o possuem conhecimento nem mesmo familiariza��o com as ferramentas de coleta. Esta � uma quest�o muito pertinente e recorrente do meio. H� evid�ncias \cite{Silvertown2009,Silvertown2011,Cohn.2008} que estes dados produzidos por cidad�os comuns possam sim ser confi�veis. Para isto � necess�rio que algumas medidas sejam seguidas.

Para alguns tipos de projetos, \citeonline{Cohn.2008} defende que os volunt�rios devem possuir algum tipo de treinamento b�sico, para que os dados sejam coletados conforme o solicitados pelos cientistas, assim diretrizes devem ser definidas. Parte dessas diretrizes devem limitar o trabalho do volunt�rio, especificando um determinado foco de coleta, por exemplo. Esta especifica��o evita diversos ru�dos nos dados e ao comparar a coleta feita entre os volunt�rios, os dados seguir�o a mesma sem�ntica, facilitando a verifica��o de erros. H� relatos que projetos anteriores baseados em ci�ncia cidad�, possuiam resultados variados por causa destes, ao inv�s de dados exatos \cite{Cohn.2008}. Solicitar que os volunt�rios desempenhem trabalhos simples auxilia na exatid�o dos dados. 

\cite{Silvertown2009} enumera tr�s fatores cruciais para o aparecimento de projetos de ci�ncia cidad� nos �ltimos anos. Primeiramente, a Internet como meio de disseminar informa��es e adquirir dados do p�blico, assim como a tecnologia dos smartphones, onde mais e mais aplicativos destes utilizam diversos sensores para coletar diferentes dados. Segundo fator se deve aos cientistas profissionais perceberem que volunt�rios s�o uma fonte sem custos de trabalho e habilidades pessoais como tamb�m de poder computacional, projetos que requerem adquirir diversos dados ao longo do globo, necessitam de ajuda, podendo ser de volunt�rios, para se obter sucesso. Terceiro fator, grandes fundadores de projetos cient�ficos procuram beneficiar projetos que utilizam cientistas cidad�os em seus trabalhos.

Com o aparecimento destes novos projetos, a ci�ncia cidad� moderna pode ser classificadas em tr�s formas, conforme o n�vel de envolvimento do volunt�rio e a tecnologia utilizada no projeto (ver figura \ref{fig:pt_citizen_science}).

\begin{figure}[htb]
    \centering
    \includegraphics[width=\textwidth,height=\textheight,keepaspectratio]{figuras/pt_citizen_science.png}
    \caption{A ci�ncia cidad� pode ser classificadas em tr�s formas: computa��o volunt�ria, onde os volunt�rios doam o poder computacional de suas m�quinas para pesquisas cient�ficas; pensamento volunt�rio utilizado em projetos que requerem a cogni��o dos voltunt�rios, como classifica��o de imagens; e sensoriamento volunt�rio onde a capta��o de dados � crucial para ter uma pesquisa.}
    \label{fig:pt_citizen_science}
\end{figure}

Estas formas ser�o introduzidas a seguir.

\section{Computa��o Volunt�ria}
\label{ch:computacao_voluntaria}

Recentemente, o n�mero de projetos que se beneficiam de ci�ncia cidad� est� aumentando, cada dia h� novos projetos surgindo. Estes projetos chamados de ci�ncia cidad� moderno est�o se tornando frequente por causa da accebilidade das tecnologias atuais, o que n�o requer aparatos especializados para realizar as pesquisas. Como mencionado anteriormente, as informa��es que \citeonline{Darwin1859} utilizou em suas pesquisas foram enviadas atrav�s de cartas que demoravam muitos meses. Hoje, com o avan�o da internet, volunt�rios em diferentes partes do mundo podem fornecer diferentes tipos de dados a uma pesquisa. Seja por dispositivos m�veis, que uma vez ligados a internet podem fornecer dados de qualquer lugar, ou ent�o por meio de computadores. 

Na d�cada de 80, a internet ainda era apenas um embri�o e poucos tinham acesso. Existiam menos de 200,000 servidores espalhados no mundo. At� ent�o n�o existiam p�ginas para serem navegadas, a principal forma de troca de mensagem era atrav�s de e-mail, criado em 1977 \cite{HistoryOfInternet:David,HistoryOfInternet:Anthony}, outros meios eram por telnet e IRC, apenas. S� no in�cio da d�cada de 90 que as primeiras p�ginas de internet foram criadas, ap�s a defini��o do \textit{WWW}\footnote{World-Wide Web} criado por \citeonline{berners1992world}. A internet estava tornando-se popular, com seus aproximados 1 milh�o de servidores e suas 50 p�ginas de internet.

Com a populariza��o da internet, iniciou-se a apari��o de projetos not�veis da ci�ncia cidad� moderna. Os computadores desta �poca eram caros e possuiam pouco poder computacional, apenas grandes ind�strias e universidades tinham acesso a m�quinas de grande desempenho. Neste per�odo, surgiram os primeiros projetos de computa��o volunt�ria, onde os volunt�rios doavam o tempo de processamento ocioso de suas m�quinas a projetos que necessitavam de grande poder computacional para trabalhar em cima dos seus dados. O conceito desta forma de projeto era de dividir a grande massa de dados existente em pequenas por��es que fossem poss�veis para os volunt�rios efetuar downloads, visto que naquela �poca n�o havia internet de banda larga. A forma de conex�o � internet ainda era discada e o modem mais r�pido deste per�odo era o de 56k\footnote{56 kilobits por segundo} \cite{AndersonCKLW02}.

No meio da d�cada de 90, surgiu os projetos \textbf{\textit{GIMPS}}\footnote{\textit{\url{http://www.mersenne.org/}-Great Internet Mersenne Prime Search}} e \textbf{\textit{Distributed.net}}\footnote{http://www.distributed.net/}\cite{Anderson1999,AndersonCKLW02,Hayes1998}.

\textit{GIMPS} foi primeiro projeto de computa��o volunt�ria de grande porte a ser realizado, \textit{GIMPS}, tinha como objetivo encontrar n�meros primos de Mersenne, nome dado em homenagem ao estudioso Marin Mersenne da teoria dos n�meros. A formula destes n�meros equivale  $ M_n = 2^n -1 $, onde $n$ � um n�mero natural. O desafio de descobrir n�meros de Mersenne est� diretamente ligado ao fato destes n�meros serem expon�nciais, tendo assim milhares de d�gitos em sua composi��o. At� 1996, in�cio do projeto \textit{GIMPS}, apenas 34 n�meros primos de Mersenne eram conhecidos, logo no primeiro ano do projeto foram descubertos mais dois  n�meros, $ M_{1398269} $ e $M_{2976221}$. O primeiro n�mero possui $852.365$ algoritmos, o segundo $1.814.262$, uma opera��o que seria imposs�vel para ser realizada por uma pessoa. Ambos foram descobertos na primeira vers�o do software disponibilizado pelo projeto, sendo calculado por um computador Pentium 90 MHz e Pentium 100MHz, respectivamente. Hoje, h� o conhecimento de 48 n�meros de Mersenne, sendo o �ltimo n�mero descoberto o $M_{57885161}$ com $17.425.170$ d�gitos em 2013, desde o in�cio do projeto foram descobertos 14 destes n�meros \cite{Marsenne:Primes,Hayes1998}.

A \textit{Distributed.net}, lan�ado em 1997, tinha o principal objetivo de quebrar a criptografia gerada pela empresa RSA, para o desafio \textit{RSA Secret-Key Challenge} que correspondia a uma chave de 56-bit e possuia uma recompensa de $10.000,00$ d�lares. O projeto criado por Earle Ady e Christopher G. Stach II, contava com a colabora��o de mais de $300.000$ volunt�rios utilizando o tempo de seus computadores realizando For�a Bruta\footnote{For�a Bruta � a forma de tentar obter a a resposta de uma senha atrav�s de �numeras tentativas.} em cima de parte do c�digo disponibilizado para o desafio. Em 250 dias a chave foi descoberta, utilizando um poder computacional equivalente a 26 mil \textit{Pentium 200}. Outro desafio com uma chave de 64-bit tamb�m foi conclu�do ap�s 4 anos e o pr�mio pago. A empresa de seguran�a RSA, havia dito que para quebrar uma chave de 64-bit, seria necess�rio mais de 100 anos testando todas as possibilidades e combina��es. Atualmente o projeto est� focado em quebrar uma chave de 72-bit \cite{distributed.net:online,Distributed.net:wired}. 

Em 1999, \citeonline{AndersonCKLW02} iniciou um dos primeiros e bem sucedido projeto de ci�ncia cidad�, o objetivo deste projeto era de encontrar vida inteligente no espa�o, atrav�s da an�lise de sinais de r�dios captadas do espa�o, \textit{SETI@Home}. Por�m para conseguir analizar esses sinais, o projeto contou com o uso de diferentes computares, todos espalhados pela internet formando um grande sistema distru�do de processamento. Os volunt�rios que se cadastravam no site podiam fazer download de um aplicativo que s� era ativado quado o computador estava em modo ocioso, o aplicativo recebia pacotes de sinais a serem analizados e no fim do processamento enviava os resultados obtidos ao servidor do projeto.

\section{Pensamento Volunt�rio}
\label{ch:pensamnto_voluntario}

\cite{Anderson1999}, a frente do projeto \textit{SETI@Home}, iniciou o desenvolvimento de uma nova ferramenta para diminuir as barreiras que ele havia encontrado ao longo do desenvolvimento de seu projeto, viabilizando assim novas iniciativas para utilizar computa��o volunt�ria de forma r�pida e sem grandes conhecimentos de computa��o. Outra fun��o da ferramenta, BOINC\footnote{Berkeley Open Infrastructure for Network Computing}, era de avaliar a exatid�o e veracidade dos dados antes de envi�-los aos servidores dos projetos \cite{anderson2003public}. Esta ferramenta j� foi utilizada por mais de 150 projetos, tendo atualmente 70 projetos online. Estes fatos s� foram alcan�ados pela acessibilidade que novos projetos do tipo de computa��o volunt�ria tiveram com a cria��o da nova ferramenta e tamb�m pela visibilidade que a ferramenta deu aos projetos deste porte.

No fim da d�cada de 90, \citeonline{dinucci1999fragmented} cunhou o termo \textit{Web 2.0} dizendo que a internet at� ent�o n�o tinha mostrado o seu real potencial. As p�ginas eram simplesmentes recursos est�ticos onde a navega��o era composta de uma simples requisi��o a este recurso e o recurso era ent�o exibido na tela dos computadores da �poca. A revolu��o da \textit{Web 2.0} seria marcada pela interatividade do conte�do, permitindo a qualquer pessoa utilizando de um computador ou dispositivo m�vel, n�o mais carregar um simples recurso est�tico, mas sim o poder de interagir com este recurso, expressando ideias e adicionando novas informa��es a \textit{Web}. Diversas novas ferramentas foram criadas nesta nova era. Diferentes tipos de \textit{blogs}, \textit{wikis} e p�ginas de internet repletos de conte�dos din�micos.

Al�m da ferramenta \textit{BOINC} realizar verifica��es redundantes para melhorar a exatid�o dos resultados, esta tamb�m era uma plataforma de pontua��o. Na p�gina da ferramenta existem diversas estat�sticas dos projetos em andamento, estas estat�sticas s�o atualizadas din�micamente conforme os resultados s�o submetidos pelos volunt�rios. Uma vez que os novos resultados s�o submetidos e aprovados, cada volunt�rio tem o valor da sua contribui��o ao projeto calculado novamente. Tendo assim um sistema de creditos, que pontua a participa��o dos volunt�rios, destacando os que mais contribuem. Este � considerado o sistema de recompensa dos usu�rios, levando o projeto a ter cada vez mais volunt�rios contribuindo com a performance da sua m�quina para ganhar notoriedade \cite{Anderson1999}.

Com a possibilidade de conte�do din�mico e a intera��o dos usu�rios, sugiram novos tipos de projetos que passaram a utilizar a capacidade cognitiva dos volunt�rios para analisar visualmente determinados dados e em fun��o destes tomar a��es. 

Um dos primeiros projetos de pensamento volunt�rio foi criado para detectar pequenas part�culas de poeira interestelar coletada pela mss�o \textit{Stardust} da NASA, lan�ado em 1999 \cite{Stardust:Mission}. Andrew Westpahl, o idealizador do projeto \textit{Stardust@Home} teve a ideia de utilizar a percep��o dos usu�rios para substituir a falta de uma tecnologia de reconhecimento de padr�o capaz de encontrar as part�culas m�nusculas coletada pela miss�o. Westpahl, estima que levaria mais de um s�culo para sua equipe poder atingir o objetivo do projeto sem a ajuda dos volunt�rios. Para encontrar as part�culas, foi disponibilizado no in�cio do projeto em 2006, 1.6 milh�es de imagens na p�gina do projeto, estas imagens recriam a experi�ncia que um cientista exerceria se estivesse analisando as amostras atr�s de particulas atrav�s de um microsc�pio, estabelecendo uma melhor imagem ajustando o seu foco. Estas 1.6 milh�es de imagens, foram feitas para simular esta fun��o, chamado de ``filmes de foco'', onde diversas imagens foram feitas de um determinado ponto mas utilizando posi��es diferentes para se ter uma melhor nitidez da imagem ou n�o. � fun��o do volunt�rio verificar se esta imagem esta bem focada, verificando a nitidez das imagens disponibilizadas e se h� alguma particula presente em alguma destas imagens\cite{Hand2010}. Caso n�o exista nenhuma imagem focada, o volunt�rio deve informar atrav�s do site esta situa��o, assim como outras diveras situa��es que podem acontecer. Para conhecer essas determinadas situa��es, os volunt�rios necessitam realizar um treinamento antes de iniciar o seu trabalho de volunt�rio efetivamente, neste treinamento h� dicas e orienta��es de como proceder. Este tipo de treinamento de qualifica��o � muito comum em projetos de ci�ncia cidad� \cite{Silvertown2009,Anderson1999}.

Apesar do projeto \textit{Stardust@Home} possuir o sufixo \textit{@Home}, este n�o utiliza a ferramenta BOINC, idealizada por \citeonline{anderson2003public}. Este sufixo � apenas uma homenagem aos projetos de ci�ncia cidad�, efetuado por estes. Contudo, \citeonline{Anderson1999} estudou o projeto e idealizou uma nova ferramenta para tornar projetos de pensamento volunt�rio mais comuns, assim como o uso da ferramenta BOINC. BOSSA\footnote{\url{http://boinc.berkeley.edu/trac/wiki/BossaIntro}} � um \textit{middleware} respons�vel por dividir o trabalho em tarefas menores e atribu�-las aos volunt�rios, realizando verifica��es do n�vel de acertividade das respostas dadas pelos volunt�rios atrav�s de redund�ncia das tarefas. 

Em 2007, iniciou-se o projeto \textit{GalaxyZoo} com o objetivo de classificar imagens de galaxias. As imagens captadas por um telesc�pio rob�tico, \textit{Sloan Digital Sky Survey}\footnote{http://www.sdss.org/} era apresentadas para os volunt�rios e estes haviam de decidir se a imagem continha alguma gal�xia, se houvesse, qual era a sua forma eliptica ou espiral, se fosse esta �ltima, ainda havia uma �ltima pergunta, qual o sentido da sua rota��o\cite{Hand2010}. Pelo n�mero de imagens que deveriam ser classificadas os cientistas envolvidos acreditavam que iria demorar mais de 2 anos para que todas as imagens fossem classificadas\cite{GalaxyZooAbout}. Por�m, com apenas um dia de funcionamento, o \textit{GalaxyZoo} conseguiu reunir 35 mil volunt�rios que fizeram aproximadamente 1.5 milh�es de classifica��es. \citeonline{Raddick2009b} compara o resultado obtido em um dia com a de um aluno de gradua��o que em uma semana conseguiu classificar apenas 50 mil gal�xias. O projeto classificou perto de um milh�o de gal�xias utilizando-se de mais de 150 mil volunt�rios. Em sua segunda vers�o, o \textbf{GalaxyZoo} contou com mais de 200 mil volunt�rios para classificar de forma mais detalhada 300 mil gal�xias previamente classificadas \cite{willett2013galaxy}. 

\textit{Rosseta@Home} outro importante projeto de ci�ncia cidad�. Quando criado em 2005, este projeto seguia a mesma tend�ncia do \textit{Seti@Home} assim como os demais projetos da fam�lia \textit{@Home}, ser mais uma computa��o volunt�ria. O objetivo deste projeto � de utilizar o grande poder computacional reunido para prever o enovelamento de prote�nas, um processo qu�mico em que a estrutura de uma prote�na assume a sua configura��o funcional, podendo assim desenvolver novas prote�nas para combater diversas doen�as. Como os demais projetos da fam�lia, os volunt�rios tinham que executar o instalador do projeto utilizando o BOINC e este s� iria entrar em funcionamento quando o computador estivesse ocioso, em modo de protetor de tela. Por�m, diversos emails foram enviados ao projeto de volunt�rios que queriam contribuir mais, dizendo ser poss�vel obter melhores resultados se eles pudessem interagir com o modelo de prote�na que estava aparecendo ali no protetor de tela. 

Com isto em mente, David Baker desenvolveu um novo projeto de pensamento volunt�rio com caracter�sticas de um jogo atrav�s da internet, onde os volunt�rios s�o os jogadores e precis�o achar as melhores solu��es para os desafios, utilizando o enovelamento de prote�nas \cite{Hand2010}. \textit{Foldit}, lan�ado em 2008, provou que os volunt�rios conseguem realizar um melhor trabalho do que um computador. Este � um dos primeiros trabalhos que envolve tanto computa��o volunt�ria quanto o pensamento volunt�rio \cite{Cooper2010}.

Um dos resultados mais significativos, foi alcan�ado pelos jogadores do projeto \textit{Foldit} em 2011, onde os volunt�rios resolveram um problema proposto pelos pesquisadores ``jogando'' em apenas 3 semanas. Os cientistas lan�aram o desafio a partir do instante em que os m�todos autom�ticos come�aram a n�o retornar bons resultados, ao analisar os resultados dos jogadores, observaram que estes eram suficientemente bons para encontrar uma r�pida solu��o \cite{Khatib2011}.

\citeonline{mcgonigal2011reality} sugere que os problemas atuais poderiam ser solucionados efetivamente de outra forma, atrav�s dos jogos. Concientizando pessoas dos problemas reais e inserindo os diversos problemas como contexto dos jogos, os jogadores iriam buscar solu��es para estes desafios. Diversos pesquisadores sugerem que construir projetos de ci�ncia com a tem�tica na forma de jogos poderiam atrair ainda mais os volunt�rios e resolver ainda mais r�pido os desafios empregados \cite{Silvertown2009,Hand2010,Khatib2011,Anderson1999,Pereira:2013:NoInUs,Cooper2010,Mansell2012}.

\section{Sensoriamento Volunt�rio}
\label{ch:sensoriamento_voluntario}

As duas formas de ci�ncia cidad� discutidas anteriormente tem suas vantagens conforme a necessidade do m�todo de pesquisa cient�fica que ir� ser executada. Computa��o volunt�ria tem maior vantagem para as atividades cient�ficas que requerem grande poder computacional, avaliando grandes quantidades de dados, o que levariam horas para uma pessoa realizar. J� o pensamento volunt�rio, mostra-se mais eficaz em atividades que requerem o poder cognitivo dos volunt�rios, como detec��es de padr�es utilizando inspe��o visual.

Algumas vezes, os dados dispon�veis para projetos cient�ficos n�o s�o suficientes e a coleta de outros tipos de medidas se faz necess�ria. Adquirir novas fontes de dados n�o � um simples trabalho, necessita-se de equipes capazes de efetuar as coletas cient�ficas conforme padr�es exigidos e ferramentas adequadas para tal.
A realiza��o desta tarefa, por�m, se feita apenas pelos cientistas envolvidos diretamente com o projeto, pode se tornar custosa e demorada. Portanto, uma alternativa vi�vel � a utiliza��o de sensoriamento volunt�rio onde volunt�rios contribuem com dados, sejam anota��es de observa��es, capta��o de imagem ou sons de um ambiente. A tecnologia atual permite solu��es de menor custo para coletar diversos tipos de dados para serem utilizados em projetos cient�ficos.

O sensoriamento volunt�rio � uma jun��o de Informa��o Geogr�fica Voluntariada (IGV), termo de \citeonline{Goodchild2007} para descrever as tarefas realizadas por volunt�rios ao fornecer dados complementares a determinadas localiza��es geogr�ficas com a finalidade de enriqucer a base de dados, com sensoriamento m�vel \cite{Lane2010}, onde pesquisas s�o conduzidas atrav�s de aparelhos m�veis dos volunt�rios utilizando os sensores dispon�veis neste. 

\subsection{Informa��o Geogr�fica Voluntariada}

\citeonline{Goodchild2007} categoriza de \textbf{informa��o geogr�fica voluntariada} o fen�meno que vem atraindo cidad�os de toda parte, a cria��o de informa��es geogr�ficas. Tarefa que antigamente era de uso exclusivo das ag�ncias oficiais respons�veis por realizar cartografias dos pa�ses, por�m com o avan�o da tecnologia, qualquer pessoa pode ter acesso a ferramentas de edi��o de mapa online e colaborar com alguma informa��o geogr�fica. Seja esta espec�fica sobre um determinado pr�dio, dado sua latitude e longitude, como tamb�m por vastos terrenos, ao descrever uma cidade. Este recente paradigma representa uma dram�tica inova��o e certamente ter� grandes impactos aos sistemas de informa��o geogr�fica.

Este conceito de cria��o de dados geogr�ficos por volunt�rios vem sendo amplamente estudado por diferentes frentes. Para a ind�stria, o desenvolvimento de plataformas baseado em web onde os usu�rios possam enviar seus dados. Para o governo, o conceito � estudado para poss�veis aplica��o em sistemas de alerta a surtos de doen�as e monitoramento constante para verificar os impactos ambientais locais causados por mudan�as clim�ticas globais. Para pesquisadores acad�micos em como adaptar ambientes de sistemas de informa��o geogr�ficos para utilizar, armazenar e analisar os dados coletados por volunt�rios \cite{elwood2008volunteered}.

Um marcante projeto de informa��o geogr�fica voluntariada, \textit{Wikimapia}, � refer�ncia nesta categoria, onde volunt�rios podem contribuir com novas informa��es geogr�ficas simplesmente escolhendo um local (atrav�s de uma coordenada geogr�fica) e complementando a informa��o local \cite{Wikimapia:online}. Possui um objetivo ambicioso de descrever todo o globo terrestre com o m�ximo de inform��es geogr�ficas ut�is reunidas atrav�s do uso de volunt�rios, organiz�-las e disponibiliz�-las para diversos outros usos p�blicos das informa��o. Atrav�s de uma interface simples, figura \ref{fig:wikimapia}, para que os volunt�rios que n�o possuam experi�ncias com edi��o de mapas possam tamb�m, de forma r�pida e intuitiva, colaborar com o projeto. Foi lan�ado em maio de 2006 como uma ferramenta similar a wikipedia, onde todos podem editar seu conte�do, e em pouco tempo diversos voltunt�rios j� haviam se cadastrado, com menos de 3 meses o projeto possuia 1 milh�o de novos registros geogr�ficos criados apenas por volunt�rios. Em julho de 2012 o projeto j� atingia a marca de 19 milh�es de registros geogr�ficos criados \cite{Wikimapia:history}.

\begin{figure}[htb]
    \centering
    \includegraphics[width=\textwidth,height=\textheight,keepaspectratio]{figuras/wikimapia.png}
    \caption{Modo de inser��o de dados do portal wikimapia.}
    %\FONTE{\citeonline{NASA}}
    \label{fig:wikimapia}
\end{figure}

Outro projeto que merece destaque nesta categoria � o renomado \textit{OpenStreetMaps}. Muito parecido com a ferramenta de mapas do Google, mas de forma livre, o OSM possui diferentes caracter�sticas e funcionalidades. Lan�ado em julho de 2004 e deste ent�o vem atraindo novos volunt�rios a cada dia, tendo um crescimento constante (ver figura \ref{fig:osm_table}). Constru�do para que volunt�rios mantenham sempre os dados geogr�ficos atualizados, assim como o wikimapia, por�m o OSM conta diversas funcionalidades para integra��o dos seus dados a projetos de terceiros, utilizando-se de protocolos abertos e com muitas op��es para exportar os dados. 

\begin{figure}[htb]
    \centering
    \includegraphics[width=\textwidth,height=\textheight,keepaspectratio]{figuras/osm_table.PNG}
    \caption{Gr�fico do crescimento mensal de usu�rios registrados e contribui��es realizadas ao projeto \textit{OpenStreetMap}.}
    \FONTE{\citeonline{haklay2008openstreetmap}}
    \label{fig:osm_table}
\end{figure}

\subsubsection{Sistemas de Alertas}

Estudos sugerem que este novo conceito de explorar e criar informa��es geogr�ficas podem levar a um novo paradigma de cria��o de sistemas de alertas, considerando que cada pessoa como um sensor, h� cerca de 6 bilh�es de sensores no globo \cite{elwood2008volunteered,Goodchild2007,Gouveia2004,Gouveia2008}. Geralmente, no segundo momento ap�s um desastre como os ocasionados por furac�es e tsunamis, � dif�cil de se ter informa��es sobre os locais que foram afetados por in�meras raz�es, seja por falta de eletricidade, um bom campo de vis�o para se obter imagens ou equipamentos de comunica��o. Por�m a popula��o das �reas afetadas conhecem suficientemente bem o local e poderia reportar ou auxiliar atividades como de resgate atrav�s de dispositivos m�veis utilizando mensagens, imagens e voz. \citeonline{Goodchild2007} reportar que um sistema deste n�vel h� de existir nos pr�ximos anos. Em seu trabalho \citeonline{Schade2011} conclui que dados geogr�ficos obtidos por volunt�rios podem ser complementares aos dados de sensoriamento remoto.

\subsection{Sensoriamento M�vel}
A era p�s-PC, mencionado pela primeira vez por \citeonline{DavidClark2013:bio}, em que estimava que os computadores deixariam de ser o principal dispositivo eletr�nico est� se tornando realidade. Os computadores pessoais est�o sendo descentralizados, como a d�cadas muitos j� previam. Sua utilidade est� mais fardada a um \textit{hub} tecnol�gico, servindo de meio de comunica��o com outros dispositivos m�veis como smartphones e tablets. 

No in�cio dos anos 90, os celulares eram realidades para pouco, mas diversos fatores alteraram a rota deste para a sua populariza��o. O avan�o da tecnologia permitiu reduzir o tamanho dos componentes eletr�nicos e sensores f�sicos digitais, tamb�m a quantidade de energia requerida por esses. 

Diversos tipos de sensores est�o cada vez mais presentes em celulares, tornando dispositivos m�veis em unidades de coleta de informa��o de grande precis�o. C�meras e sensores de localiza��o j� s�o frequentes na maioria dos celulares, estes permitem obter a sua localiza��o em qualquer parte do globo terrestre atrav�s de \textit{GPS}\footnote{\textbf{G}lobal \textbf{P}ositioning \textbf{S}ystem � um sistema de navega��o baseado em uma constela��o de 24 sat�lites.}, sendo alguns modernos integrado com \textit{GLONASS}\footnote{\textbf{GLO}bal \textbf{NA}vigatsionnaya \textbf{S}putnikovaya \textbf{S}istema, um sistema de navega��o russo, constitu�do por uma constela��o de 21 sat�lites}. H� tamb�m dispositivos que apresentam sensores de proximidade, girosc�pio, magenet�metro, aceler�metro, bar�metro e de luz ambiente. Mas sempre h� um sensor presente nos celulares, o microfone que em conjunto com os demais sensores supracitados torna-se uma poderosa ferramenta de monitoramento ambiental.

Em recentes relat�rios, \citeonline{Gartner2014} aponta que em 2015 os dispositivos m�veis (tablets, celulares e smartphones) ir�o ultrapassar os computadores pessoais (computadores de mesa e port�teis), em quantidade, conforme figura \ref{fig:gartner}.

\begin{figure}[htb]
    \centering
    \includegraphics[width=\textwidth,height=\textheight,keepaspectratio]{figuras/gartner2014.png}
    \caption{Relat�rio Gartner julho/2014}
    \FONTE{\citeonline{Gartner2014}}
    \label{fig:gartner}
\end{figure}

Com esta vis�o, novos projetos tem surgido com a inten��o de utilizar os sensores presentes nos dispositivos m�veis como fonte de dados cient�ficos, coletando constantemente informa��es do dia a dia de quem os utilizam \cite{Lane2010,Burke2006}. 



Por anos a comunidade acad�mica e industrial vem debatendo o uso de dispositivos m�veis em pesquisas de sensoriamento, por�m sem grandes avan�os at� datas recentes. \citeonline{Lane2010} atribui esta mudan�as aos seguintes fatores:

\begin{enumerate}
    \item Sensores embarcados - utilizados primeiramente como forma de melhorar a experi�ncia de uso para os usu�rios, como aceler�metro, encontraram novas formas de uso e chamaram a ate��o de pesquisadores. Diversos sensores novos est�o revolucionando as pesquisas, como GPS e bar�metro.
    \item Program�veis - H� uma cole��o infinita de documenta��o na internet de como programar para dispositivos m�veis de terceiros. As grandes plataformas de dispositivo m�veis, como Apple, Google, e Windows Phone possuem documenta��es detalhadas, permitindo qualquer pessoa com um pouco de conhecimento de programa��o aprender a linguagem e desenvolver aplicativos.
    \item Lojas de aplicativo - Os desenvolvedores de aplicativos utilizam o servi�o de loja de aplicativo do fabricante correspondente para publicar sua nova cria��o. Permitindo alcan�ar diferentes tipos de usu�rios em toda a parte.
    \item Computa��o na nuvem - Utilizando-se de computa��o na nuvem, os dispositivos m�veis podem armazenar dados e at� efetuar c�lculos atrav�s de servidores na internet, sem a necessidade de utilizar estas funcionalidades apenas local, proporcionando um grande crescimento de uso e descentraliza��o de dados.
\end{enumerate}


\chapter{ForestWatchers}
\label{ch:forestwatchers}

O projeto ForestWatchers (\url{http://www.forestwatchers.net}) prop�e o desenvolvimento e o lan�amento de uma iniciativa de ci�ncia cidad� com o objetivo de envolver e integrar cidad�os ao redor do planeta na tarefa de monitorar o desmatamento das florestas tropicais \cite{ForestWatchersDesc}. Estes cidad�os poder�o de suas casas, por meio de uma interface \textit{Web}, inspecionar imagens recentes de sat�lite de �reas de florestas. Estas podem ser de uma reserva ind�gena na Amaz�nia, uma floresta nacional em Born�u ou um parque em Queensland. As imagens s�o ent�o classificadas em �reas de floresta ou n�o-floresta, por meio de um algoritmo de classifica��o supervisionado pelos volunt�rios na \textit{Web}. Conforme mencionado por \citeonline{Ipeirotis2010}, erros e at� mesmo fraude podem ser automaticamente tratados pela redund�ncia do sistema. Para isso, � necess�rio atrair e manter um grande n�mero de volunt�rios \cite{Soares:2011:EmCiSc}. Estima-se que cem mil volunt�rios analisando uma �rea de 100.000 hectares cada, com um fator de redund�ncia de 20, podem examinar uma �rea de 500 milh�es de hectares, cerca de 40\% a 50\% da �rea estimada das florestas tropicais do mundo \cite{ForestWatchersDesc}.

O projeto conta com desenvolvedores do Laborat�rio Associado de Computa��o e Matem�tica Aplicada (LAC) do INPE, do \textit{Citizen Cyberscience Centre} (CCC), e do Departamento de Ci�ncia e Tecnologia (DCT) da Universidade Federal de S�o Paulo (UNIFESP), com apoio do \textit{Open Society Foundations} (OSF), \textit{United Nations Institute for Training and Research} (UNITAR), e \textit{UNITAR's Operational Satellite Application Programme} (UNOSAT).

A seguir, ser� discutida a metodologia empregada no projeto.

\section{Metodologia}
\label{ch:metodologia}

A metodologia usada neste projeto � inspirada no bem-sucedido programa de detec��o de desflorestamento DETER do INPE. Assim, como no sistema DETER, o projeto ForestWatchers tamb�m utiliza imagens do sensor MODIS, com resolu��o de $250$ metros (por�m qualquer outro sensor de sat�lite que forne�a suas imagens gratuitamente pode ser utilizado). Para que essas imagens possam ser exibidas para os volunt�rios � necess�rio que um pr�-processamento seja feito. 

Um diagrama ilustrativo da metodologia utilizada pelo projeto ForestWatchers pode ser visto na Figura \ref{fig:estrutura_atual}.
\begin{figure}[htb]
\includegraphics[width=\textwidth,height=\textheight,keepaspectratio]{figuras/arquitetura_atual.png}
\caption{A metodologia utilizada pelo projeto ForestWatchers.}
\label{fig:estrutura_atual}
\FONTE{\citeonline{ForestWatchersDesc}}
\end{figure}

Primeiramente � necess�rio adquirir as imagens da NASA referentes � �rea de interesse do projeto. Para esse processo, ferramentas como FTP\footnote{\textit{File Transfer Protocol (FTP)} � um protocolo para transfer�ncia de arquivo utilizado na Internet para efetuar \textit{downloads} e \textit{uploads} de arquivos.}e WGET\footnote{WGET � um programa livre para efetuar \textit{download} de conte�dos na Internet.} s�o utilizadas para realizar os downloads necess�rios. A pr�xima etapa, envolve recortar as imagens que n�o s�o pertinentes � �rea de interesse, descartando-as e consolidando as imagens restantes num �nico arquivo GeoTIFF de 16 bits. Essa etapa pode ser executada rapidamente com o aux�lio da ferramenta MODIS Reprojection Tool (MRT) \cite{MRToolManual2010}, um software gratuito disponibilizado pela NASA.

As imagens de 16 bits s�o convertidas para 8 bits por meio de um \textit{script} em Python\footnote{Python � uma linguagem de programa��o aberta.}. Logo ap�s, uma �nica imagem � consolidada utilizando-se tr�s bandas (infravermelho m�dio, vermelho, infravermelho pr�ximo, equivalentes a vermelho, verde e azul) da imagem de 8 bits. Assim, essa imagem pode ser enviada para um servidor gerenciador de arquivos de mapa. Nesse projeto utiliza-se o MapServer (\url{http://www.mapserver.org}), respons�vel por tratar as requisi��es de inser��o e sele��o das imagens georreferenciadas, e de retornar apenas parte da imagem desejada na forma de \textit{tiles}. Todos os arquivos relacionados � imagem t�m suas informa��es extra�das no formato GeoJSON\footnote{\textit{Geographic JavaScript Object Notation} � um formato para codificar variados tipos de estruturas geogr�ficas.}, para facilitar a comunica��o entre os outros m�dulos. O algoritmo de classifica��o (ainda em desenvolvimento) faz a segmenta��o das imagens, classificando-as como  �reas de floresta e de n�o-floresta. Existindo a necessidade de supervis�o das imagens, ser�o criadas tarefas para os usu�rios poderem classific�-las visualmente.

O \citeonline{PyBossa2013} � o sistema respons�vel por gerenciar a cria��o e distribui��o das tarefas automaticamente, conforme necess�rio. Esse � um sistema livre que permite um usu�rio criar e gerenciar projetos que requeiram cogni��o humana, tais como classifica��o de imagem, transcri��o e geo-codifica��o. Esse sistema � baseado no \citeonline{Boinc2008}, plataforma online desenvolvida para facilitar a cria��o e a opera��o de projetos baseados em ci�ncia cidad�. Essa nova implementa��o traz maiores benef�cios em rela��o ao sistema original, Bossa, por ser desenvolvida em Python e possuir uma API\footnote{\textit{Application Programming Interface} (API) � um protocolo com o objetivo de servir como interface para os componentes de softwares, permitindo comunicarem entre si.} RESTful \cite{Richardson2008}. 

Com as tarefas criadas, os volunt�rios podem classificar as imagens de forma ordenada. O projeto, com o uso do sistema de redund�ncia que envia a mesma tarefa para diferentes volunt�rios, garante um aumento na confiabilidade dos resultados \cite{Ipeirotis2010}.

\chapter{Metodologia}
\label{ch:metodologia}

O conceito de um m�dulo de sensoriamento volunt�rio consiste b�sicamente de \textbf{(a)} um dispositivo capaz de coletar dados, \textbf{(b)} estrutura de recebimento e armazenamento, e \textbf{(c)} um sistema de visualiza��o dos dados obtidos \cite{VolunteerSensing2011,Gouveia2008}. 

Para a valida��o deste conceito o seguinte m�dulo foi constru�do em duas etapas. Um aplicativo h�brido, utilizado pelos volunt�rios para enviarem os dados coletados a partir de um dispositivo m�vel. E uma infraestrutura tecnol�gica, divida em: uma camada de recebimento e armazenamento, processamento e visualiza��o dos dados.

Este cap�tudo descreve a metodologia utilizada em cada etapa da constru��o do m�dulo, por fim, detalha o experimento de coleta de dados utilizado por este m�dulo.

\section{Aplicativos} % (fold)
\label{sub:aplicativos}

% At� o momento, a aplica��o m�vel � a �nica forma que os volunt�rios t�m para submeter dados de coleta ao projeto.
Supondo que um vetor de coleta de dados de dif�cil acesso, com alta complexibilidade e desenvolvido com o �nico prop�sito de coletar dados, constru�dos sobre demanda, sejam fatores inibidores, restringindo significantemente o n�mero de poss�veis volunt�rios comitidos com o projeto \cite{Lane2010}. Optou-se por utilizar uma aplica��o desenvolvida para dispositivos m�veis (smartphones e tablet). Hoje, a grande maioria possui um dispositivo m�vel com capacidade de conectar-se a internet, tirar fotos, captar �udio e gravar v�deos. Com a populariza��o dos sensores embargados em dispositivos m�veis e as expectativas de crescimento deste mercado \cite{Lane2010,Gartner2014}, utiliz�-los como ferramenta para projetos de ci�ncia cidad� parecem bastante sugestivas.

Foram criados quatro aplicativos como prot�tipos para testar suas funcionalidades verificarem a sua signific�ncia em rela��o as necessidades do projeto: (i) dois destes aplicativos utilizando-se sistemas prontos ou de f�cil uso para a cria��o da ferramenta de coleta de dados. H� ferramentas dispon�veis gratuitamente para auxiliar a constru��o de aplicativos para dispositivos m�veis. Estes, em grande parte, seguem o princ�pio aberto-fechados \cite{meyer1988:open_closed}, onde uma pessoa com conhecimento pode estender as funcionalidades principais, mas n�o modific�-las diretamente; (ii) um utilizando bibliotecas nativas, disponibilizadas pelos fabricantes de dispositivos m�veis para criar novos aplicativos, como estes que s�o f�cilmente encontrados nas lojas online de aplicativos; e por fim, (iii) um utilizando uma abordagem de desenvolvimento de aplicativos h�brido, onde h� uma biblioteca em comum entre os demais dispositivos existentes e necessitando de apenas uma linguagem de programa��o.

A metodologia para o desenvolvimento dos prot�tipos (i), (ii) e (iii) s�o descritos nas se��es \ref{ssub:aplicativo_com_sistemas_prontos}, \ref{ssub:aplicativo_com_biblioteca_nativa} e \ref{ssub:aplicativo_com_biblioteca_hibrida}, respectivamente.

\subsection{Aplicativo com Sistemas Prontos} % (fold)
\label{ssub:aplicativo_com_sistemas_prontos}

Existem alguns aplicativos que permitem criar ferramentas de coleta de dados com sensores utilizando formul�rios feitos atrav�s de interfaces gr�ficas e inserido no aplicativo. Estes sistemas prontos s�o indicados para efetuar a cria��o aplicativos voltado para pesquisas ou coleta de dados de uma forma r�pida, uma vez que estas possuem toda uma infraestrutura para armazenamento de dados e interface gr�fica para a constru��o dos formul�rios.

EpiCollect � um destes sistemas e foi utilizado para construir o primeiro prot�tipo do projeto.
Foi desenvolvido um formul�rio baseado nos principais requisitos do projeto no qual visa captar �udio, v�deo e imagem juntamente com a coordenada geogr�fica atual do volunt�rio, conforme este observa um fen�meno.

Este sistema � de muito f�cil uso, bastando ter acesso a internet para cadastrar-se e criar o formul�rio. 

\subsection{Aplicativo com Biblioteca Nativa} % (fold)
\label{ssub:aplicativo_com_biblioteca_nativa}

\subsection{Aplicativo com Biblioteca H�brida} % (fold)
\label{ssub:aplicativo_com_biblioteca_hibrida}

\section{Infraestrutura Tecnol�gica} % (fold)
\label{sub:infraestrutura_tecnologica}

\subsection{Armazenamento} % (fold)
\label{sub:armazenamento}

\subsection{Servidor para Internet} % (fold)
\label{sub:servidor_web}

\subsubsection{Back-End} % (fold)
\label{ssub:back_end}

\subsubsection{Front-End} % (fold)
\label{ssub:front_end}


\section{Coleta de Dados} % (fold)
\label{sub:coleta_de_dados}






\chapter{Resultados e Discuss�es}
\label{ch:resultados}

\section{Conclus�es e Trabalhos Futuros}
\label{ch:conclusoes}
%% \include{./docs/introducao}
%% \chapter{REVIS�O BIBLIOGR�FICA}

\section{Sistemas de Monitoramento de Desmatamento}
\subsection{PRODES}
\subsection{DETER}

\section{Ci�ncia Cidad�}

Ci�ncia cidad� n�o � novidade. O ato de fazer pesquisas cientificas utilizando-se da colabora��o de diversos indiv�duo vem dos tempos mais remotos \todo[cite]{� preciso inserir cita��o?}. Um grande exemplo de ci�ncia realizada atrav�s de colabora��o em massa, A Origem das Esp�cies. \citeonline{darwin-origin-of-species-1859}, contou com a ajuda de mais de 2000 colaboradores entre esses, especialistas, bi�logos e pesquisadores de diferentes �reas, tendo tamb�m a colabora��o de pessoas leigas.  O s�tio \citeonline{DarwinProject_Correspondents} reune mais de 7.500 das cartas que Darwin manteve com seus correspondentes durante sua pesquisa. 

As cartas naquela �poca demoravam meses para serem recebidas e lidas,  o tr�mite de responder uma carta e obter um novo retorno desta, chegava a levar quest�es de anos para acontecer. Tudo isto devido a este meio de comunica��o n�o ter as mesmas tecnologias atuais, tornando a tarefa de entregar uma carta hoje em dia simples e comum , um trabalho dificultoso e lento. Naquela �poca, os servi�os de correspond�ncias era feito por mensageiros a p�, a cavalo ou atrav�s de charretes, tendo n�vios � vela para correspond�ncias pelos mares. \citeonline{Hyde1891} relembra que para a determinada �poca, estes servi�os eram caros e n�o acess�vel para todos, o que dificultava ainda mais o compartilhamento de ideias. Em 1840, com a grande reforma brit�nica de postagem,  \textit{Penny Postage} \footnote{A reforma realizada pela \textit{Royal Mail} do Reino Unido cobrava apenas um \textit{Penny}, menor moeda do sistema monet�rio da �poca, para entregar as cartas indiferente da dist�ncia.}, houve uma maior difus�o e uso dos servi�os, elevando o envio de cartas de 82.500.000 para 169.000.000 em um ano, mais do dobro.

A refer�ncia cl�ssica relatada nos trabalhos de ci�ncia cidad� � sobre o projeto \cite{Projeto das Aves}\todo[cite]{Encontrar a cita��o do projeto de Aves}. Iniciado como um desafio em 1900, o auch... vem at� hoje colaborando com a ci�ncia mantendo um invent�rio anual de pass�ros atrav�s das observa��es de diversos volunt�rios no Estados Unidos. Este projeto deu origem a diversas pesquisas, que atrav�s dos dados por esse coletado, puderam mapear a fauna, a biodiversidade e a migra��o de diversas esp�cies. Por ainda estar em vig�ncia, volunt�rios de toda a parte dos estados unidos se engajam na �poca de dezembro para procurar e observar os diferentes pass�ros . Hoje, o centro de pesquisa que deu origrm a este projeto � considerado um dos centros especializados quando o assunto � projeto em ci�ncia cidad� voltado para a �rea biol�gica.

Ci�ncia cidad� pode ser aplicada nas diversas �reas da ci�ncia, ultimamente vem demonstrando ser bastante eficaz em atividades nas �reas de biologia, astr�nomia e desenvolvimento urbano. 

Exemplos de sucesso nestas �reas s�o:
\begin{enumerate}
	\item[Foldit] - Uma aplica��o de biologia onde volunt�rios ganham pontos a cada nova enzima descuberta. \todo[Foldit]{Adicionar mais coment�rios}
	\item[ZooGalaxy] - Com o objetivo de descubrir novas estrelas, prop�e-se a mostrar imagens do espa�o aos volunt�rios e estes classific�-las. Considerado um dos mais famosos pelo n�mero de volunt�rios atingido e grandes resultados. \todo[ZooGalaxy]{Adicionar mais coment�rios? E resultados?}
	\item[Seti@Home] - Considerado um dos primeiros projetos de ci�ncia cidad� dos tempos modernos, definitivamente um marco na hist�ria de como fazer ci�ncias.  Voltado a descobrir algum tipo de son no espa�o que pudesser ser considerado como emitido de alguma forma n�o humana, o Seti@Home utilizava o computador dos volunt�rios para processar grandes volumes de dados para buscar seu objetivo. 
\end{enumerate}

\subsection{Internet e Web 2.0}



No fim da d�cada de 90, \citeonline{dinucci1999fragmented} cunhou o termo \textit{Web 2.0} dizendo que a internet at� ent�o n�o tinha mostrado o seu real potencial. As p�ginas eram simplesmentes recursos est�ticos onde a navega��o era composta de uma simples requisi��o a este recurso e o recurso era ent�o exibido na tela dos computadores da �poca. \citeonline{dinucci1999fragmented} ainda nota que at� ent�o n�o havia diferen�as de dispositivos, a mesma p�gina que era carregada e exibida em um monitor de 17'' era a mesma a ser exibida em uma tela de 2'' de um PDA\footnote{Personal Digital Assistant - PDA}. A revolu��o da \textit{Web 2.0} seria marcada pela interatividade do conte�do, permitindo a qualquer pessoa utilizando de um computador ou dispositivo m�vel, n�o mais carregar um simples recurso est�tico, mas sim o poder de interagir com este recurso, expressando ideias e adicionando novas informa��es a \textit{Web}.




\subsection{Defini��o}
\subsection{Hist�rico}
\subsection{Avan�os na Ci�ncia}
\subsection{Suas Formas}
\subsubsection{Projetos}

\section{Sistemas de Monitoramento Ambientais Colaborativos}
\subsection{Projetos}

\section{ForestWatchers}
\subsection{Proposta}
\subsection{Metodologia}
\subsection{Resultados}

\section{Aplicativos M�veis}
\subsection{Surgimento}
\subsection{Divis�o de Mercado}
\subsection{Aplicativos Nativos}
\subsection{Aplicativos H�bridos}




%% %% ++ Desenvolvimento do Aplicativo
%% ++ Capta��o dos Dados
\chapter{METODOLOGIA}
\section{Desenvolvimento do Aplicativo}
\section{Capta��o dos Dados}
%% \include{./docs/resultado_e_discussoes}
\bibliography{./bib/publicacao_nova} %% aponte para seu arquivo de bibliografia no formato bibtex (p.ex: referencia.bib)
\inicioIndice
\include{./docs/contracapa}
\todos
\end{document}